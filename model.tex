% vim: set spell:
% vim: set spelllang=en:
\documentclass[a4paper,11pt]{article}
\usepackage[utf8]{inputenc}
\usepackage{fancyhdr}
\usepackage{hyperref}
\usepackage{amsmath}
\usepackage[T1]{fontenc}


\author{Raphael Catolino}
\title{Numerical solution of restricted n-body problems, using
the central differences approximation.}

\begin{document}
\maketitle

\begin{abstract}
  In order to simulate the trajectories of celestial bodies,
  we want a numerical solution to the n-body problem created
  by the modeling of the gravitational interactions between each body.

  We want to be able to simulate these trajectories for a space scale
  with the same order of magnitude as the solar system, and a time scale
  varying from a few days to a few years.

  We use a simple finite difference method on a discrete time scale
  to approximate the second order derivative component in the formulation
  of Newton's second law. No consideration has been given to the collision case.

  Instead of focusing on precision or accuracy, the main goal of the
  study is the ease of implementation. Therefore we only study here the
  stability and convergence of the solution.
\end{abstract}
\newpage
\tableofcontents
\newpage

\section{Introduction}
The gravitational interaction between two bodies, not accounting for relativistic
effects, can be be expressed using Newton's law of universal gravitation.
In a system consisting of only those two bodies, their relative trajectories
can then be easily computed using Newton's second law of motion.

However, in a system where an arbitrary number of bodies
interact with each other, solving the ODE system describing the
trajectories becomes non-trivial. It has been shown that no analytical solution
could be found for the general three body problem\footnote{Bruns H.E., 1887}.
Several ways have been designed to numerically solve this problem for an
arbitrary number of bodies (the n-body problem). Here we use the central difference
approximation to express the acceleration of a body at a certain time $t$ as a
function of its postilion at the times $t-\delta t$, $t$ and $t+\delta t$.

\section{Problem formulation}
Let there be $n$ bodies $B_i;\quad i \in [[1, n]]$ in a Euclidian space and $m_i$
their respective mass.
Let $\vec{R_i}(t);\quad i \in [[1,n]]$ be their respective positions in the associated Cartesian
coordinate system at the time $t \ge 0$.\\
For two bodies $B_i$ and $B_j;\quad i,j \in [[1, n]]^2$ the force applied by $B_i$ upon
$B_j$ is given by Newton's law of universal gravitation :
\begin{equation}
  \vec{F_{i/j}} = G \frac{m_i m_j}{|{\vec{R_i} - \vec{R_j}}|^3} (\vec{R_i} - \vec{R_j})
\end{equation}

Moreover, following Newtown's law of motion, the acceleration of $B_j$ is :
\begin{equation}
  \forall j \in [[1, n]]:\quad m_j \frac{d^2\vec{R_j}}{dt^2} =
  \mathop{\sum_{i=1}}_{i \ne j}^n \vec{F_{i/j}}
\end{equation}

Therefore, the acceleration of every body in the system is described by the following system :
\begin{equation}
  \forall j \in [[1, n]]:\quad \frac{d^2\vec{R_j}}{dt^2} =
  G*\mathop{\sum_{i=1}}_{i \ne j}^n \frac{m_i}{|{\vec{R_i} - \vec{R_j}}|^3} (\vec{R_i} - \vec{R_j})
  \label{odesystem}
\end{equation}

We want to solve this system for all $\vec{R_j}; \quad i \in [[1, n]]$.

\section{Discretization of the time dimension and Central differences approximation}
Let $\delta t$ be an arbitrarily small, positive, time interval and $t_0$ an initial time.
Furthermore let's define $\forall k \in \mathbf{N}:\quad \vec{R_{j,k}} = \vec{R_j}(t_0+k\delta t)$.\\
Thus, using the $2^{nd}$ order central difference approximation we obtain :
\begin{equation}
  \forall k \in \mathbf{N*},\quad \forall j \in [[1, n]]:\quad
  \frac{d^2\vec{R_{j,k}}}{dt^2} = \frac{\vec{R_{j,k+1}} - 2 \vec{R_{j,k}} + \vec{R_{j,k-1}}}{\delta t^2}+ O(\delta t^2)
\end{equation}

Therefore in the discretized time dimension, \eqref{odesystem} becomes :
\begin{multline}
  \forall k \in \mathbf{N*},\quad \forall j \in [[1, n]]:\\
  \frac{\vec{R_{j,k+1}} - 2 \vec{R_{j,k}} + \vec{R_{j,k-1}}}{\delta t^2} = 
  G*\mathop{\sum_{i=1}}_{i \ne j}^n \frac{m_i}{|{\vec{R_{i,k}} - \vec{R_{j,k}}}|^3}
  (\vec{R_{i,k}} - \vec{R_{j,k}})
\end{multline}
Thus we can express each the position of each body at a certain time step $k \ge 2$
as a function of the position of all the bodies at the time steps $k - 1$ and $k - 2$ :
\begin{multline}
  \forall k \in \mathbf{N-\{0,1\}},\quad \forall j \in [[1, n]]:\\
  \vec{R_{j,k}} = 2 \vec{R_{j,k-1}} - \vec{R_{j,k-2}} +
  \delta t^2 G* \mathop{\sum_{i=1}}_{i \ne j}^n m_i
  \frac{\vec{R_{i,k-1}} - \vec{R_{j,k-1}}}{|{\vec{R_{i,k-1}} - \vec{R_{j,k-1}}}|^3}
  \label{dodesystem}
\end{multline}
\section{Initial value problem}
In order to use \eqref{dodesystem} to compute the positions of the bodies, we need their
initial positions for the first two time steps, for $k = 0$ and $k = 1$.
We will use the initial position and speed of the bodies and derive their position
for the time step $k = 1$ from it.\\
Let's note $\vec{R_{j0}}$ and $\vec{V_{j0}}$ the
position and speed vectors for $k=0$ and \linebreak[4] $j \in [[1, n]]$.
Substituting the speed by it's forward difference approximation we obtain :
\begin{equation}
  \forall j \in [[1, n]]: \quad
  \vec{V_{j0}} = \frac{R_{j,1} - R{j, 0}}{\delta t} + O(\delta t)
  \Leftrightarrow
  \vec{R_{j,1}} = \vec{R_{j0}} + \delta t \vec{V_{j0}} + O(\delta t)
\end{equation}
It's worth noting that using this approximation causes the leading error to go
down from $O(\delta t^2)$ to $O(\delta t)$.
\section{Stability and convergence study}
\section{Adaptive step size}
\section{Conclusion and results}

\end{document}

